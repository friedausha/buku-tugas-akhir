\chapter{KESIMPULAN DAN SARAN}
Bab ini membahas tentang kesimpulan yang didasari oleh hasil uji coba yang telah dilakukan pada bab sebelumnya. Kesimpulan nantinya sebagai jawaban dari rumusan masalah yang dikemukakan. Selain kesimpulan, juga terdapat saran yang ditujukan untuk pengembangan penelitian lebih lanjut di masa depan.

\section{Kesimpulan}
Dalam pengerjaan Tugas Akhir ini setelah melalui tahap perancangan aplikasi, implementasi metode, serta uji coba, diperoleh kesimpulan sebagai berikut:

\begin{enumerate}
	\item Untuk mengimplementasi \textit{transfer learning} untuk mendeteksi daun digunakan library Keras dan pengimplementasian ekstraksi fitur menggunakan model yang tersedia.
	\item Untuk mengembangkan aplikasi pendeteksi daun berbasis Android yang mengirim gambar ke server dan menerima balasan berupa jenis daun diterapkan REST API untuk memanggil server yang kemudian akan mengirim balasan ke Android. Untuk memanggil server digunakan Retrofit2.
	\item Untuk melakukan proses klasifikasi daun pada server, digunakan model yang telah disimpan. Model dibuat dengan menggunakan \textit{transfer knowledge} dari model-model yang sudah disediakan Keras.
	\item Arsitektur yang paling akurat dalam mendeteksi daun menggunakan Logistic Regression adalah MobileNet, dengan akurasi 96\%.
\end{enumerate}

\section{Saran}
Berikut merupakan beberapa saran untuk pengembangan sistem di masa yang akan datang. Saran-saran ini didasarkan pada hasil perancangan, implementasi, dan uji coba yang telah dilakukan.
\begin{enumerate}
	\item Diperlukan hardware yang mumpuni untuk melatih model.
	\item Memperbaiki desain pada aplikasi Android, termasuk pemilihan warna dan tata letak.
	\item Membuat tombol "Bantuan" untuk mempermudah pengguna.
\end{enumerate}