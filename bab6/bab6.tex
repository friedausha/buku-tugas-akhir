\chapter{KESIMPULAN DAN SARAN}
Bab ini membahas tentang kesimpulan yang didasari oleh hasil uji coba yang telah dilakukan pada bab sebelumnya. Kesimpulan nantinya sebagai jawaban dari rumusan masalah yang dikemukakan. Selain kesimpulan, juga terdapat saran yang ditujukan untuk pengembangan penelitian lebih lanjut di masa depan.

\section{Kesimpulan}
Dalam pengerjaan Tugas Akhir ini setelah melalui tahap perancangan aplikasi, implementasi metode, serta uji coba, diperoleh kesimpulan sebagai berikut:

\begin{enumerate}
	\item Arsitektur yang paling akurat dalam mendeteksi daun menggunakan Logistic Regression adalah MobileNet.
	\item Pada metode transfer learning, hampir semua arsitektur mendapatkan akurasi diatas 90\% kecuali ResNet50.
	\item Dataset daun tidak cocok jika fiturnya diekstraksi dengan ResNet50, yang menyebabkan akurasi kurang dari 60\%.
	\item Arsitektur MobileNet juga merupakan arsitektur paling ringan, dengan ukuran 91 MB dibandingkan arsitektur lain yang sampai ratusan MB.
\end{enumerate}

\section{Saran}
Berikut merupakan beberapa saran untuk pengembangan sistem di masa yang akan datang. Saran-saran ini didasarkan pada hasil perancangan, implementasi, dan uji coba yang telah dilakukan.
\begin{enumerate}
	\item Diperlukan hardware yang mumpuni untuk melatih model.
	\item Jika ingin menggunakan ResNet mungkin dapat dicoba dengan ResNet yang layernya lebih banyak seperti ResNet152.
	\item Memperbaiki desain pada aplikasi Android, termasuk pemilihan warna dan tata letak.
	\item Membuat tombol "Bantuan" untuk mempermudah pengguna.
\end{enumerate}