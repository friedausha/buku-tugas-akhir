\chapter{DAFTAR NOTASI}
\begin{tabularx}{\linewidth}{c X}
	$ \mathbb{Z} $ & Himpunan bilangan bulat. \\
	$ \mathbb{Z}_n $ & Himpunan bilangan bulat positif hingga $ n $ eksklusif. \\
	$ \mathbb{Z}_n^* $ & Sebuah \textit{multiplicative group} dengan himpunan beranggotakan bilangan bulat hingga $ n $ eksklusif. \\
	$ \text{ord}(g) $ & Order sebuah bilangan $ g $, yaitu berapa kali $ g $ harus dioperasikan dengan $ g $ agar menghasilkan elemen identitas. \\
	$ \phi(n) $ & \textit{Euler Totient Function} atau \textit{Euler Phi}. Menotasikan banyaknya nilai yang koprima dengan $ n $. \\
	$ \omega $ & Sebuah bilangan pada $ \mathbb{Z}_n^* $ dengan order sebesar $ h $ dimana $ h < \phi(n) $. \\	
	$ \mathbb{H}_{(a, n)} $ & Himpunan yang anggotanya merupakan elemen $ \mathbb{Z}_n $. Himpunan ini berisi seluruh nilai yang mungkin dibangun dari $ a^i\ (mod\ n) $ untuk seluruh nilai $ 0 \leq i < n $. \\
	$ \mathbb{H}_{(\omega, n)} $ & Himpunan yang anggotanya merupakan elemen $ \mathbb{Z}_n $. Himpunan ini berisi seluruh nilai yang mungkin dibangun dari $ \omega^i\ (mod\ n) $ untuk seluruh nilai $ 0 \leq i < \frac{\phi(n)}{h} $ dimana $ ord(\omega) = h $. \\
	$ [k]_n $ & Sebuah kelas residu $ k $ dengan modulus $ m $, yaitu himpunan $ {k + in\ :\ i \in \mathbb{Z}} $. \\
	$ p_{\text{turtle}} $ & Sebuah \textit{pointer} suku sebuah deret yang dibangun oleh suatu random function $ f_n(x) $. \textit{Pointer} suku ini bergerak satu langkah setiap iterasinya. \\
	$ p_{\text{hare}} $ & Sebuah \textit{pointer} suku sebuah deret yang dibangun oleh suatu random function $ f_n(x) $. \textit{Pointer} suku ini bergerak setiap $ 2^i $ iterasi untuk $ i $ yang terus bertambah.
\end{tabularx}