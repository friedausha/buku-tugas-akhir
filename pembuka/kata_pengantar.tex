\chapter {KATA PENGANTAR}

Puji syukur penulis panjatkan kepada Tuhan Yang Maha Esa atas penyertaan dan karunia-Nya sehingga penulis dapat menyelesaikan tugas akhir dan laporan akhir dalam bentuk buku ini. Penelitian tugas akhir ini dilakukan untuk mengeksplorasi topik yang menarik perhatian penulis serta memenuhi salah satu syarat dalam mendapatkan gelar Sarjana Komputer di Departemen Informatika Fakultas Teknologi Informasi dan Komunikasi Institut Teknologi Sepuluh Nopember. Penulis memiliki harapan bahwa apa yang penulis kerjakan dapat membawa manfaat bagi perkembangan ilmu pengetahuan di bidang komputer dan bagi penulis sendiri selaku peneliti.

Penulis ingin mengucapkan terima kasih kepada semua pihak yang telah membimbing dan memberi dukungan baik secara langsung maupun tidak langsung selama proses pengerjaan tugas akhir ini maupun selama menempuh masa studi. Pihak tersebut antara lain:

\begin {enumerate}
	\item Tuhan yang Maha Esa, yang dengan izin dan kasihnya penulis bisa kuliah di Informatika ITS dan menyelesaikan perkuliahannya.
	\item Keluarga penulis yang telah memberikan dukungan moral, doa, dan material sehingga penulis dapat menyelesaikan Tugas Akhir ini.
	\item Bapak Dwi Sunaryono S.Kom., M.Kom. dan Bapak Dr. Radityo Anggoro S.Kom, M.Sc.  selaku pembimbing dari Tugas Akhir penulis, yang dengan bantuan dan arahan beliau penulis dapat menyelesaikan buku ini.
	\item Irsyad Rizaldi, atas dukungan dan pertemanannya selama penulis kuliah.
	\item Keluarga Pratama Ristyantika, yang terus memotivasi penulis untuk menyelesaikan tugas akhir dan menjadi teman yang dapat diandalkan.
	\item Valdi, Ade, Ipul, dan Petrus yang memberi penulis wawasan tentang dasar-dasar pembelajaran mesin.
	\item Pi, partner bridge penulis yang darinya penulis belajar banyak hal.
	\item GLBK dan Petualang, yang menginspirasi penulis atas keseimbangan antara prestasi dan kehidupannya.
	\item Seluruh dosen dan karyawan Departemen Informatika Fakultas Teknologi Informasi dan Komunikasi Institut Teknologi Sepuluh Nopember yang telah memberi ilmu dan waktunya untuk mempersiapkan penulis agar siap untuk masuk ke dalam dunia kerja. 
\end {enumerate}

Penulis menyadari bahwa laporan Tugas Akhir ini masih memiliki banyak kekurangan. Oleh karena itu dengan segala kerendahan hati penulis mengharapkan kritik dan saran dari pembaca untuk perbaikan penulis kedepannya. Selain itu, penulis berharap laporan Tugas Akhir ini dapat berguna bagi pembaca secara umum.

\begin{flushright}
Surabaya, Juni 2019 \\*
\vspace{5em}
\penulis
\end{flushright}